
\section{Appendix Ion Balance}
When dissolving compounds in water, most of them dissociate in separate ions. In natural water, significant concentrations of the following ions are present:
\begin{itemize}
    \item \textbf{Cations} $Na^+, K^+, Ca^2+, Mg^2+, Fe^2+, Mn^2+, NH_4^+$
    \item \textbf{Anions} $CI^-, HCO_3^-, NO_3^-, SO_4^2-$
\end{itemize}
When more than one compound has dissolved in water, it is typically not possible to trace the origin of the mix. Because water is electrically neutral, the sum of the valence concentration of the cations is equal to that of the anions:
\[ \underset{Cations}\sum \frac{c_i}{M_i*z_i} =
\underset{Anions}\sum \frac{c_i}{M_i*z_i}\]
where
\begin{tabbing}
$c_i$ = concentration in g/mL of ion\\
$M_i$ = molar mass in g/mol of ion\\
$z_i$ = valency of the element
\end{tabbing}

One can use this to deduce the concentration of a cation or anion if the rest of the ions are known. To take the calculation of the valence concentration of sodium as an example:
\[Na^+_{meq/L} = \underset{Anions}\sum \frac{c_i}{M_i*z_i} - \underset{Cations-Na^+}\sum \frac{c_i}{M_i*z_i}\]

