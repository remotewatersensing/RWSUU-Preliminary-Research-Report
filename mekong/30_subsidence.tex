\newpage
\subsection{Subsidence}
Subsidence is a general term for downward vertical movement of the Earth's surface, which can be caused by both natural processes and human activities. 

Subsidence can be measured by 

From 2014-2019, Utrecht University launched a project called Rise and Fall that aims to enhance the capabilities of individuals and organisations to develop sustainable strategies for dealing with mainly groundwater extraction and land subsidence in the increasingly urbanising Mekong Delta. Throughout the years, they have generated several reports describing the issue of subsidence at hand. To describe the issue of subsidence in the Mekong Delta, it is best to take a look at this project's findings. \cite{riseandfall}

\subsubsection{Findings}

One of major findings of the project is that the elevation of the Mekong Delta is much lower than formerly estimated. By means of a high accuracy Digital Elevation Model generated using InSAR technology \cite{insar} it could be demonstrated that the average elevation of the delta is only 0,8 meter above mean sea level, instead of 2,6 meter as used as basic assumption by the international research community earlier. This means that the Mekong Delta is much more vulnerable to sea level rise as generally assumed. \cite{minderhoud2017}

The project researchers also showed that land subsidence takes place very fast. The Mekong Delta is subsiding at a rate of 1-5 centimetres a year, much higher than sea level rises following global warming. From a farmer’s perspective, this means that saline tidal water from the coast can enter the delta’s rivers much further inland and affect rice production.

\subsubsection{Cause}
The extraction of groundwater for drinking water, agriculture and fisheries in the Mekong Delta is the most probable cause of the dramatic land subsidence in this low-lying area. 

Researchers modelled the expected subsidence based on the amount of groundwater extraction and compared this expected subsidence for the actual subsidence. For the Mekong Delta, about 75\% of the cases of measured subsidence is at least matched by the best estimated modelled subsidence. \cite{minderhoud2017}

Land subsidence from ground water extraction occurs when large amounts of groundwater have been withdrawn from certain types of rocks, such as fine-grained sediments. The rock compacts because the water is partly responsible for holding the ground up. When the water is withdrawn, the rocks falls in on itself.

\subsubsection{Mitigation}
The top-down way groundwater governance is organised in Vietnam makes it difficult to define and implement local interventions. Still, designating specific areas for sedimentation is a suggested strategy to encourage elevation-building with nature in deltas. \cite{minderhoud2020}

One could also look into other ways to extract water, such as surface water treatment. While these actions won't solve subsidence, it delays future relative sea-level rise, giving the Mekong Delta time to adapt.