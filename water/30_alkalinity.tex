\newpage
\subsection{Alkalinity}
Total alkalinity is a measurement of the water’s ability to resist a reduction in pH. Alkalinity ions resist reduction in pH by attracting a Hydrogen ion if needed. Total alkalinity is measured by the concentration (parts per million) in the water. \cite{standardmethods} In this section, a closer look is taken at total alkalinity and how it affects the water quality in different scenario's.

\subsubsection{Irrigation Water}
In order to evaluate the quality of irrigation water, one should know both pH and alkalinity. A pH test by itself is not an indication of alkalinity. Water with high alkalinity always has a pH value 7 or above, but water with high pH doesn't always have high alkalinity. 

Irrigating with water having a high pH causes no problems as long as the alkalinity is low (0 to 100 ppm calcium carbonate). The irrigation water will probably have little effect on growing medium pH because it has little ability to neutralize acidity.

Problems occur when water that has both high pH and high alkalinity is used for irrigation. The pH of the growing medium may increase significantly with time, decreasing crop rates. It is much more difficult to predict the effects of irrigating outdoor flower crops, gardens, and landscape plants with water having high pH and high alkalinity. 

Moderately alkaline water could be beneficial however as a source of extra $Ca$ and $Mg$ for crops prone to $Ca$ and $Mg$ deficiencies \cite{umassalkalinity}

\subsubsection{Water for aquatic life}
Fish and other aquatic life generally need a pH range of 6.5 to 8.0. \cite{fishphghkh} Since alkalinity buffers against rapid pH changes, the alkalinity helps protect the living organisms who need a specific pH range. Higher alkalinity levels in surface water can buffer acid rain and other acid wastes. This can prevent pH changes that are hazardous to aquatic life. 

%For drinking water?

\subsubsection{Sensors}

While there does exist a total alkalinity (ppm) sensor, it exists in the form of a lab-on-chip prototype developed by the national oceanography centre. \cite{alkalinitysensor} It is a bulky 6kg device meant to be used in large ships, and in no form suitable to be deployed in this project.

\paragraph{Lab-on-Chip Total Alkalinity Sensor}\mbox{} \cite{alkalinitysensor}
\begin{table}[h!]
	\centering
	\adjustimage{height=4cm,valign=c}{water/31_alkalinitysensor.jpg}\quad
	\begin{tabular}{| l | l |}
    \hline
    Measurement Range & 600umol/kg \\
    Measurement Accuracy &  5 umol/kg \\
    Response time & 12 min \\
    Supply Voltage & 10-16V \\
    Weight & 6kg \\
    Availability & Unknown \\
    \hline
	\end{tabular}
\end{table}